%-------------------------------------------------------------------------------
%	SECTION TITLE
%-------------------------------------------------------------------------------
\cvsection{Publications}


%-------------------------------------------------------------------------------
%	CONTENT
%-------------------------------------------------------------------------------
\begin{cventries}

%---------------------------------------------------------
  \cventry
    {\textbf{Da-Wei Li}, Danqing Huang, Tingting Ma, Chin-Yew Lin} % Organization
    {\href{https://www.microsoft.com/en-us/research/publication/towards-topic-aware-slide-generation-for-academic-papers-with-unsupervised-mutual-learning/}{Towards Topic-Aware Slide Generation For Academic Papers With Unsupervised Mutual Learning}} % Title
    {AAAI 2021 (CCF A)} % Location
    {May. 2021} % Date(s)
    {
      \begin{cvitems} % Description(s) of tasks/responsibilities
        \item {Generating slides from papers by extractive summarization techniques and unsupervised mutual learning.}
        \item {During mutual learning, two extractors (a Log-Linear Classifier and a Neural Sentence Selection Model) learn collaboratively by teaching each other and bootstrap their performances.}
      \end{cvitems}
    }

%---------------------------------------------------------
  \cventry
    {Huifan Yang, \textbf{Da-Wei Li}, Zekun Li, Donglin Yang, Bin Wu} % Organization
    {\href{https://www.easychair.org/publications/preprint_open/jz3j}{Open Relation Extraction with Non-Existent and Multi-Span Relationships}} % Title
    {KR 2022 (CCF B)} % Location
    {Feb. 2022} % Date(s)
    {
      \begin{cvitems} % Description(s) of tasks/responsibilities
        \item {Designed Query-based Multi-head Open Relation Extractor (QuORE) to extract single/multi-span relations and detect non-existent relationships.}
        \item {QuORE is a multi-head framework construct with two sub-modules, the SSE (Single-Span Extraction) and the QASL (Query-based Sequence Labeling) which share loss during training and a selector to choose which result to use during inference.}
      \end{cvitems}
    }

%---------------------------------------------------------
  \cventry
    {Huifan Yang, \textbf{Da-Wei Li}, Zekun Li, Donglin Yang, Jinsheng Qi, Bin Wu} % Organization
    {\href{https://www.easychair.org/publications/preprint_open/lLmV}{Open Relation Extraction via Query-Based Span Prediction}} % Title
    {KSEM 2022 (CCF C)} % Location
    {May. 2022} % Date(s)
    {
      \begin{cvitems} % Description(s) of tasks/responsibilities
        \item {Designed Query-based Open Relation Extractor (QORE), a multilingual Transformers-based language model to derive a representation of the interaction between arguments and context.}
        \item {Experiments including Multilingual ORE, ablation study on non-query/query-based and non-LM/LM models, zero-shot domain transferability, and few-shot learning ability.}
      \end{cvitems}
    }

%---------------------------------------------------------
\end{cventries}
