%-------------------------------------------------------------------------------
%	SECTION TITLE
%-------------------------------------------------------------------------------
\cvsection{Engineering Project}


%-------------------------------------------------------------------------------
%	CONTENT
%-------------------------------------------------------------------------------
\begin{cventries}

%---------------------------------------------------------
  \cventry
    {Side Project} % Organization
    {Raspberry Pi 集群} % Title
    {} % Location
    {Sep. 2018} % Date(s)
    {
      \begin{cvitems} % Description(s) of tasks/responsibilities
        \item {我利用 4 个 Raspberry Pis 来搭建集群。自制了一个能快速部属 Hadoop、Spark 等 ecosystem 的脚本。}
      \end{cvitems}
    }

%---------------------------------------------------------
  \cventry
    {个人接案} % Organization
    {Leapsy AR 眼镜影像串流遥控摄像云台} % Title
    {} % Location
    {Jul. 2017 - Oct. 2017} % Date(s)
    {
      \begin{cvitems} % Description(s) of tasks/responsibilities
        \item {我收集来自一个搭载 Android 系统的 AR 眼镜的 sensor 数据,用以捕捉当前使用者的姿态,并将此讯号传递给 Raspberry Pi 来同步遥控摄像云台之面向,最终将影像透过 Wi-Fi 来实时串流回眼镜中。}
        \item {我设计 3D print 的模型来组合摄相头与两个伺服马达,同时也针对马达与 Raspberry Pi 设计了其电源电路。}
      \end{cvitems}
    }

%---------------------------------------------------------
  \cventry
    {本科毕业专题} % Organization
    {基于模组化架构之四轴飞行器设计及其于影像辨识之应用} % Title
    {} % Location
    {Apr. 2016 - Sep. 2016} % Date(s)
    {
      \begin{cvitems} % Description(s) of tasks/responsibilities
        \item {从零打造一台四轴飞行器,包含马达、扇页、sensor等各晶片的挑选,与整体电路的设计。目标设计理念是可快速将整体框架套用在任何开发板上(只需对 IO 进行特化)。}
        \item {在 control board 使用 PID 平衡算法,并在 CV board 上搭载基于 OpenCV 的 object detection。}
        \item {分别在 HOLTEK MCU Design Contest 2016 获得佳作与在 ARM Design Contest 2016 获得 Top 10。同时也证明了我们的设计理念。}
      \end{cvitems}
    }

%---------------------------------------------------------
\end{cventries}
