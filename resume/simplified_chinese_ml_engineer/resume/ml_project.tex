%-------------------------------------------------------------------------------
%	SECTION TITLE
%-------------------------------------------------------------------------------
\cvsection{ML/NLP Project}


%-------------------------------------------------------------------------------
%	CONTENT
%-------------------------------------------------------------------------------
\begin{cventries}

%---------------------------------------------------------
  \cventry
    {Side Project} % Organization
    {GCAKE: Graph and Context Attentional Knowledge Embedding} % Title
    {} % Location
    {Oct. 2019 - Jan. 2020} % Date(s)
    {
      \begin{cvitems} % Description(s) of tasks/responsibilities
        \item {是一个利用 self-attention 的 knowledge graph representation learning 架构。使用不仅限于三元组的信息,同时考虑了上下文与图结构之关系。}
      \end{cvitems}
    }

%---------------------------------------------------------
  \cventry
    {Side Project} % Organization
    {Similar Cases Recommendation via Legal Knowledge Graph Construction and Representation} % Title
    {} % Location
    {Aug. 2019 - Oct. 2019} % Date(s)
    {
      \begin{cvitems} % Description(s) of tasks/responsibilities
        \item {提出了一个将法律案件利用 knowledge graph embedding 进行表示并利用其来推荐之 pipeline。}
        \item {其中包含子任务有 Named-entity Recognition、Relation Extraction、Knowledge Graph Embedding。主要模型是使用 jointly-trained multitask 的 fine-tuned BERT。}
      \end{cvitems}
    }

%---------------------------------------------------------
  \cventry
    {Side Project} % Organization
    {Sentence Similarity in Intelligent Question Answering} % Title
    {} % Location
    {Aug. 2019 - Sep. 2019} % Date(s)
    {
      \begin{cvitems} % Description(s) of tasks/responsibilities
        \item {主要基于 Enhanced RCNN model (BiLSTM + Attention + CNN) 来优化句子相似度的计算。}
        \item {背景设立于智能问答之应用,以同时考量 performance 与 complexity 之间的权衡为出发点。}
      \end{cvitems}
    }

%---------------------------------------------------------
  \cventry
    {Online Course} % Organization
    {Stanford CS224n: Natural Language Processing with Deep Learning} % Title
    {} % Location
    {Jul. 2019 - Dec. 2019} % Date(s)
    {
      \begin{cvitems} % Description(s) of tasks/responsibilities
        \item {实作项目包含 Neural Dependency Parsing、Neural Machine Translation、Question Answering。}
      \end{cvitems}
    }

%---------------------------------------------------------
  \cventry
    {Kaggle Competition} % Organization
    {Jigsaw Unintended Bias in Toxicity Classification} % Title
    {} % Location
    {Feb. 2019 - May. 2019} % Date(s)
    {
      \begin{cvitems} % Description(s) of tasks/responsibilities
        \item {此竞赛之目的是要能辨别什么留言是在骂人亦或是无恶意之叙事。我们小组分别设计了多个基于不同模型(如 BERT, ELMo)的 classifier,并将其 ensemble。在竞赛期间,我们组曾达到 Top 1\%。}
      \end{cvitems}
    }

%---------------------------------------------------------
  \cventry
    {Course Project} % Organization
    {SemEval-2013 Task 13: Word Sense Induction for Graded and Non-Graded Senses} % Title
    {} % Location
    {Jun. 2019 - Jul. 2019} % Date(s)
    {
      \begin{cvitems} % Description(s) of tasks/responsibilities
        \item {这是一个 word disambiguation 的任务,考虑相同词在不同句子中可能代表不同的含意。其中有两个子任务,一个是寻找该词最相近的 WordNet 解释,另一者则是将句子间相似的词进行 clustering。}
      \end{cvitems}
    }

%---------------------------------------------------------
  \cventry
    {Course Project} % Organization
    {SemEval-2018 Task 7: Semantic Relation Extraction and Classification in Scientific Papers} % Title
    {} % Location
    {May. 2019 - Jun. 2019} % Date(s)
    {
      \begin{cvitems} % Description(s) of tasks/responsibilities
        \item {这是一个 relation classification 的任务。主要文本是来自于多篇论文之 abstract 段落。对于句子中所标示之 instance,来判断两者间是属于所给定的哪种 relation 之一。}
      \end{cvitems}
    }

%---------------------------------------------------------
  \cventry
    {Course Project} % Organization
    {Chinese Word Segmentation, Part-of-speech Tagging, Named-entity Recognition} % Title
    {} % Location
    {Apr. 2019 - May. 2019} % Date(s)
    {
      \begin{cvitems} % Description(s) of tasks/responsibilities
        \item {经典的中文 sequence labeling 任务。主要基于 BiLSTM-CRF 与其他 baselines 进行比较。}
      \end{cvitems}
    }

%---------------------------------------------------------
  \cventry
    {Digital China Innovation Contest 2019} % Organization
    {混泥土汞车砼活塞故障预警} % Title
    {} % Location
    {Jan. 2019 - Mar. 2019} % Date(s)
    {
      \begin{cvitems} % Description(s) of tasks/responsibilities
        \item {这个比赛中,数据是针对混泥土汞车在运行期间的一组 time-series 数据。目标是要预测某组序列运行后最终潜在故障的概率。我主要使用 LightGBM 并最终达到 Top 5\%。}
      \end{cvitems}
    }

%---------------------------------------------------------
\end{cventries}
