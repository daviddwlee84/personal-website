%-------------------------------------------------------------------------------
%	SECTION TITLE
%-------------------------------------------------------------------------------
\cvsection{Experiences}


%-------------------------------------------------------------------------------
%	CONTENT
%-------------------------------------------------------------------------------
\begin{cventries}

%---------------------------------------------------------
  \cventry
    {Data \& Applied Scientist} % Job title
    {微软互联网工程院 Bing Multimedia 组} % Organization
    {Suzhou, Jiangsu, China} % Location
    {Jul. 2021 - Present} % Date(s)
    {
      \begin{cvitems} % Description(s) of tasks/responsibilities
        % Core Priorities:
        \item {为 Bing、Edge、MSN 提供更好的视频推荐质量以提升 DAU 与用户的 CTR。}
        \item {在多个场景包含 MSN Article、Bing Super Caption、Edge Underside 中上线了 \textbf{Relevance Model} 来控制推荐视频之相关性以便提升更大的 coverage 同时保证推荐质量,整体降低了人工标注的 \textbf{defect rate -2.2\%}。}
        % TODO: training data amount? training strategy: negative sampling? features: text feature? offline and online measurement: precision, recall, defect rate.
        \item {在 MSN Article 的 37 个 market 中上线了 \textbf{CTR Model} 带来 \textbf{Video-CI/UU +2.4\%} 与 \textbf{Video-CI Ratio +2.33\%} 的提升。}
        % TODO: training data amount? training strategy: daily aggregation, top features: engagement feature, relevance feature? offline and online measurement: DCG, gDCG, defect rate, CI.
        \item {为 Bing VDP 提供基于 co-info 的 \textbf{CF Recall},带来 \textbf{Traffic Coverage +23.0\%} 与 \textbf{CTR/UU +1.33 \%} 的提升。}
        \item {为提升开发效率搭建线上 \textbf{Related Video Debug Tool} 与可复用的通用部件例如数据预处理模块、自动化标注流程、流程脚本视觉化等。}
      \end{cvitems}
    }

%---------------------------------------------------------
  \cventry
    {Algorithm Intern} % Job title
    {微软互联网工程院 Bing NLP 组} % Organization
    {Beijing, China} % Location
    {Jul. 2020 - Jun. 2021} % Date(s)
    {
      \begin{cvitems} % Description(s) of tasks/responsibilities
        \item {搭建 \textbf{Numeric Information Extraction System} 用于金融研报的数值信息提取. 针对任务需求设计 annotation guideline,训练 MRC-based Sequence Labeling 模型,与搭建基于 docker 的后端服务。}
        \item {搭建 \textbf{AI Writer} 的 NLP 模型用来自动生成文章与改写文章。}
        % TODO: GPT-2, Back-Translation, ...
        % \item {Built models of \textbf{Writing Assistant} which offers such as copyediting, typesetting, proofreading, indexing, page makeup.}
      \end{cvitems}
    }

%---------------------------------------------------------
  \cventry
    {Research Intern} % Job title
    {微软亚洲研究院 Knowledge Computing 组} % Organization
    {Beijing, China} % Location
    {Dec. 2019 - May. 2020} % Date(s)
    {
      \begin{cvitems} % Description(s) of tasks/responsibilities
        \item {在 AAAI 2021 上发表关于``\textbf{学术文章的简报自动生成}''的论文,并受邀在\href{https://www.bilibili.com/video/BV1fV411q7Ho/}{微软官方 Bilibili} 上进行直播分享。}
      \end{cvitems}
    }

%---------------------------------------------------------
\end{cventries}
