%-------------------------------------------------------------------------------
%	SECTION TITLE
%-------------------------------------------------------------------------------
\cvsection{Education}


%-------------------------------------------------------------------------------
%	CONTENT
%-------------------------------------------------------------------------------
\begin{cventries}

%---------------------------------------------------------
  \cventry
    {软件工程 硕士} % Degree
    {北京大学} % Institution
    {Beijing, China} % Location
    {Sep. 2018 - Jul. 2021} % Date(s)
    {
      \begin{cvitems} % Description(s) bullet points
        % \item {主要关注于 NLP 与 Knowledge Graph 相关的应用。}
        % \item {参与北大开源协会,是 ML/NLP 组的核心成员。}
        \item {主要关注于 NLP 与 Knowledge Graph 相关的应用。参与北大开源协会,是 ML/NLP 组的核心成员。}
        \item {毕业论文为《面向中文文本的数值抽取与理解方法设计与实现》,内容是设计一套符合中文特性的标注方法来训练抽取模型,并基于此模型搭建数值抽取的应用。}
      \end{cvitems}
    }

% Military Service in Taiwan during the gap year

%---------------------------------------------------------
  \cventry
    {电子工程 学士 (辅系财金)} % Degree
    {台湾科技大学} % Institution
    {Taipei, Taiwan} % Location
    {Sep. 2013 - Jun. 2017} % Date(s)
    {
      \begin{cvitems} % Description(s) bullet points
        \item {主要关注于嵌入式系统设计及其他工程类项目如 App, Web 等。}
        \item {毕业设计为《基于模组化架构之四轴飞行器设计及其于影像辨识之应用》,从零构建一台无人机,设计与调整基于 PID 的平衡算法,并结合机器视觉做物件追踪的应用展示。}
        \item {拥有 4 次个人接案经验,其中较有趣的案子为利用 Android-based 的 AR 眼镜来远程控制视频云台再将影像直播串流到眼镜中,其中涉及云台电路设计与模型3D打印。}
        \item {参加 5 次不同工程领域的竞赛,包含设计2048游戏 AI、App 设计竞赛、LED 设计竞赛与 MCU 设计竞赛等。}
      \end{cvitems}
    }

%---------------------------------------------------------
\end{cventries}
